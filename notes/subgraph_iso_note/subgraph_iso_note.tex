\documentclass[12pt]{article}

\usepackage[left=2cm,right=2cm,bottom=3cm,top=2.5cm]{geometry}

\let\proof\relax
\let\endproof\relax

\usepackage{amsmath}
\usepackage{amsfonts}
\usepackage{amsthm}
\usepackage{amssymb}
\usepackage{mathtools}
\usepackage{soul}
\usepackage{xspace}
\usepackage{xcolor}
\usepackage{enumitem}
\usepackage{hyperref}
\usepackage[capitalise]{cleveref}
\usepackage{booktabs}
\usepackage{tabularx}
\usepackage{mdframed}
\usepackage{multirow}
\usepackage{float}
\usepackage[linesnumbered,ruled,vlined]{algorithm2e}
\usepackage{algpseudocode}
\usepackage{subcaption}
\usepackage{multirow}
\usepackage{makecell}
\usepackage[all,cmtip]{xy}
\usepackage{graphicx}
\usepackage{framed}

% \pagespttyle{plain}


\setul{1ex}{.5pt}

\hypersetup{
    % pdftitle={},
    % pdfauthor={},
    colorlinks=true,
    linkcolor=black,
    citecolor=black,
    pdfstartview=FitH,
    pdfpagemode=UseOutlines,
    pdfpagelayout=OneColumn,
}

% \setcounter{tocdepth}{2}

\mdfdefinestyle{figstyle}{ %
	linecolor=black!7, %
	backgroundcolor=black!7, %
	innertopmargin=10pt, %
	innerleftmargin=25pt, %
	innerrightmargin=25pt, %
	innerbottommargin=10pt %
}

\newtheorem{theorem}{Theorem}
\newtheorem{lemma}{Lemma}
\newtheorem{hypothesis}[theorem]{Hypothesis}
\newtheorem{corollary}[theorem]{Corollary}
\newtheorem{construction}{Construction}

\theoremstyle{definition}
\newtheorem{action}{Group Action}
\newtheorem{definition}[theorem]{Definition}
\newtheorem{assumption}{Assumption}
\newtheorem{observation}{Observation}
\newtheorem{fact}{Fact}
\newtheorem{remark}[theorem]{Remark}

\newcommand{\bra}[1]{\langle{#1}|}
\newcommand{\ket}[1]{|{#1}\rangle}
\newcommand{\ip}[2]{\langle{#1}, {#2}\rangle}
\newcommand{\braket}[2]{\langle{#1}|{#2}\rangle}
\newcommand{\mode}[1]{\textnormal{\textsf{#1}}}
\renewcommand{\vec}[1]{\mathbf{#1}}
\newcommand{\F}{\mathbb{F}}
\newcommand{\K}{\mathbb{K}}
\renewcommand{\P}{\mathbb{P}}
\newcommand{\Z}{\mathbb{Z}}
\newcommand{\R}{\mathbb{R}}
\newcommand{\N}{\mathbb{N}}
\newcommand{\Matrix}{\mathrm{M}}
\newcommand{\Tensor}{\mathrm{T}}
\newcommand{\id}{\mathrm{id}}
\newcommand{\ATF}{\mathrm{ATF}}
\newcommand{\rk}{\mathrm{rank}}
\newcommand{\probsty}[1]{\textsf{#1}\xspace}
\newcommand{\GI}{\probsty{GI}}
\newcommand{\PK}{\mathrm{PK}}
\newcommand{\SK}{\mathrm{SK}}

\newcommand{\CFI}{\probsty{CFI}}
\newcommand{\ID}{\probsty{ID}}
\newcommand{\QFMI}{\probsty{QFMI}}
\newcommand{\pGpI}{$p$\probsty{GpI}}
\newcommand{\GMW}{\probsty{GMW}}
\newcommand{\Alteq}{\probsty{ALTEQ}}
\newcommand{\DATFE}{\probsty{DATFE}}
\newcommand{\MATFE}{\probsty{MATFE}}
\newcommand{\PATFE}{\probsty{PATFE}}
\newcommand{\Exp}{\probsty{Exp}}
\newcommand{\ATFA}{\probsty{ATFA}}
\newcommand{\EUFCMA}{\probsty{EUF-CMA}}
\newcommand{\sEUFCMA}{\probsty{sEUF-CMA}}
\newcommand{\EUFNMA}{\probsty{EUFNMA}}
\newcommand{\CROATFE}{\probsty{$C$-PR-psATFE-RO}}
\newcommand{\ROATFE}{\probsty{PR-psATFE-RO}}
\newcommand{\FS}{\probsty{FS}}
\newcommand{\PI}{\probsty{PI}}
\newcommand{\TIp}{\probsty{TI}}
\newcommand{\DTI}{\probsty{DTI}}
\newcommand{\TTI}{\probsty{3TI}}
\newcommand{\gainv}{\probsty{GA-Inv}}
\newcommand{\gapr}{\probsty{GA-PR}}
\newcommand{\msg}{M}
\newcommand{\cB}{\mathcal{B}}
\newcommand{\cA}{\mathcal{A}}
\newcommand{\cI}{\mathcal{I}}
\newcommand{\cX}{\mathcal{X}}
\newcommand{\cY}{\mathcal{Y}}
\newcommand{\cZ}{\mathcal{Z}}
\newcommand{\cK}{\mathcal{K}}
\newcommand{\cP}{\mathcal{P}}
\newcommand{\cS}{\mathcal{S}}
\newcommand{\cD}{\mathcal{D}}
\newcommand{\tA}{\mathtt{A}}
\newcommand{\Tr}{\mathrm{Tr}}

\newcommand{\sampcost}{\textsf{samp-cost}}
\newcommand{\colcost}{\textsf{col-cost}}
\newcommand{\invcost}{\textsf{inv-cost}}
\newcommand{\mrkcost}{\textsf{minrank-cost}}

\newcommand{\tB}{\mathtt{B}}

%\newcommand{\M}{\mathrm{M}}
\newcommand{\diag}{\mathrm{diag}}
\newcommand{\gpmul}{\circ}
\newcommand{\gpact}{\ast}
\newcommand{\class}[1]{\ensuremath{\mathrm{#1}}\xspace}
\newcommand{\NP}{\class{NP}}
\newcommand{\TI}{\class{TI}}
\newcommand{\coNP}{\class{coNP}}
%\newcommand{\coAM}{\class{coAM}} %cryptocode

%\newcommand{\secpar}{\lambda} %cryptocode
\newcommand{\usecpar}{1^\secpar}
\newcommand{\veps}{\varepsilon}
\newcommand{\bit}{\{0,1\}}


\newcommand{\FG}{\mathrm{FG}}


\newcommand{\abbrsty}[1]{\ensuremath{\mathrm{#1}}\xspace}
\newcommand{\OWA}{\abbrsty{OWA}}
\newcommand{\PRA}{\abbrsty{PRA}}
\newcommand{\PROD}{\abbrsty{PROD}}
\newcommand{\INV}{\abbrsty{INV}}
\newcommand{\QRO}{\abbrsty{QRO}}
\newcommand{\GLAT}{\abbrsty{GLAT}}

\newcommand{\pg}{\mathcal{G}}
\newcommand{\cm}{\mathcal{CM}}
\newcommand{\qdm}{\mathcal{QDM}}
\newcommand{\params}{\texttt{params}}
\newcommand{\attack}{\mathcal{A}}
\newcommand{\hvsim}{\mathcal{S}}
\newcommand{\dualkey}{{\widetilde{pk}}}
\newcommand{\signlist}{\mathcal{L}}

\newcommand{\PRF}{\mathsf{PRF}}
\newcommand{\Aut}{\mathrm{Aut}}
\newcommand{\Adj}{\mathrm{Adj}}
\newcommand{\lencom}{{\ell_{\mathrm{in}}}}
\newcommand{\lench}{{\ell_{\mathrm{ch}}}}
\newcommand{\lenr}{{\ell_{\mathrm{re}}}}
\newcommand{\prep}{\ell}

\newcommand{\algstyle}[1]{\textsc{#1}\xspace}
\newcommand{\ids}{\algstyle{ID}}
\newcommand{\gaids}{\algstyle{GA-ID}}
\newcommand{\kg}{\algstyle{KG}}
\newcommand{\dkg}{\algstyle{KG}^*}
\newcommand{\skg}{\algstyle{KeyGen}}
%\renewcommand{\sign}{\algstyle{Sign}} %cryptocode
\newcommand{\vrfy}{\algstyle{Verify}}
\newcommand{\unruhsign}{\algstyle{SIGN}}
\newcommand{\gasign}{\algstyle{GA-SIGN}}
\newcommand{\fssign}{\algstyle{FS-SIGN}}
\newcommand{\gafssign}{\algstyle{GA-FS-SIGN}}
%\newcommand{\prg}{\algstyle{PRG}} %cryptocode
\newcommand{\ggm}{\algstyle{GGM}}
\newcommand{\tr}[1]{#1^{\mathrm{t}}}
%\newcommand{\sEUFCMA}{\probsty{sEUF-CMA}}
% zhili's macros
\newcommand{\adj}{\mathrm{adj}}
\newcommand{\MEDS}{\probsty{MEDS}}
\newcommand{\LESS}{\probsty{LESS}}
\newcommand{\LCME}{\probsty{LCME}}
\newcommand{\CSIFiSh}{\probsty{CSI-FiSh}}
\newcommand{\ComSet}{\mathsf{ComSet}}
\newcommand{\ChaSet}{\mathsf{ChSet}}
\newcommand{\ResSet}{\mathsf{ResSet}}
\newcommand{\com}{\prover.\mathsf{Com}}
\newcommand{\res}{\prover.\mathsf{Res}}
\newcommand{\ver}{\verifier.\mathsf{Ver}}
\newcommand{\gen}{\mathsf{Gen}}
\newcommand{\lgen}{\mathsf{LossyGen}}
\newcommand{\Igen}{\mathsf{IGen}}
%\newcommand{\rsample}{\stackrel{\$}{\leftarrow}}
\newcommand{\ChaSp}{\mathcal{C}}
\newcommand{\rsample}{\gets_R}
\newcommand{\sgn}{\mathsf{Sign}}
\newcommand{\vrf}{\mathsf{Verify}}
\newcommand{\pkey}{\mathsf{pk}}
\newcommand{\skey}{\mathsf{sk}}
\newcommand{\ls}{\mathsf{ls}}
\newcommand{\Time}{\mathsf{Time}}
\newcommand{\Advantage}{\mathsf{Adv}}
\let\O\undefined
\let\S\undefined
\DeclareMathOperator{\O}{\mathrm{O}}
\DeclareMathOperator{\S}{\mathrm{S}}
\DeclareMathOperator{\M}{\mathrm{M}}
\DeclareMathOperator{\GL}{\mathrm{GL}}
\DeclareMathOperator{\SL}{\mathrm{SL}}

\newcommand{\enote}[1]{\textcolor{blue}{\small$\bullet$\footnote{\textcolor{blue}{Euan:
				#1}}}}

\renewcommand\qedsymbol{$\blacksquare$}

%\DeclareMathOperator{\tr}{\mathrm{tr}}


\title{
    Notes on Subgraph Isomorphism Testing and it's Linear Algebraic Analogues
}

\author{}
\date{}

\begin{document}

\maketitle

\section{Introduction}

For the sake of convenience, let $[n] = \{ 1, 2, \ldots, n \}$.

\paragraph{Graphs.} A graph is a collection of vertices and edges, formally graphs are $G=(V, E)$ where $V=[n],E\subseteq [n]{\times}[n]$. A \textit{subgraph} $G'=(V',E')$ denoted $G'\subseteq G$ is a graph such that $V' \subseteq V$ and $E'\subseteq E$.

\paragraph{Groups.} Let $G$ be a graph than $\Aut(G)$ denotes the automorphism group generated by permuting the vertices of the graph $G$. Additionally, let $S_{n}$ denote the symmetric group of order $n$. The group action of $S_{n}$ on a graph $G$ is exactly a permutation of the vertices of $G$, so the orbit $S_{n}\cdot G = \{ \sigma(G) \mid \sigma \in S_{n} \}$ is equivalent to the automorphism group of $G$.

\subsection{Graph and Subgraph Isomorphism Testing}

Two graphs are said to be \textit{isomorphic} if for two graphs $G, H$ of $n$ vertices, there exists a bijection $\phi : [n] \to [n]$, such that $(u,v)\in E(G) \implies (\phi(u),\phi(v)) \in E(H)$\footnote{Technically the definition of graph automorphism, but isomorphism can be reduced to automorphism in polynomial time by assigning an arbitrary index to each vertex of a graph}.

We can reformulate the previous statement into a algebraic problem involving group actions as follows. Given two graphs $G,H$ of $n$ vertices, if there exists a $\sigma\in \Aut(G)$ such that $\sigma(G)=H$, the two graphs are said to be isomorphic. We can even further reformulate this as deciding if the action of the symmetric group on two graphs yield the same orbits. We denote two isomorphic graphs $G\cong H$.

There has always been a close correlation between group theory and graph isomorphism testing. This leads to some interesting insights about the \textit{algebraic} structure of graph isomorphism as well as the combinatorial structure.

The \textit{Graph Isomorphism Testing Problem} asks for input graphs $G,H$ of $n$ vertices, to decided if $G\cong H$. Additionally, the \textit{Subgraph Isomorphism Testing Problem} asks for a graph $G$ of $n$ vertices and a graph $H$ of $k$ vertices with $k \leq n$, decide if there exists a subgraph $G'$ of $G$, such that $G'\cong H$.

\subsection{Applications}

In \textit{theoretical computer science} the connections between (multi)linear algebra and graph theory has been an interesting topic. Graphs are by far the most ubiquitous object within computer science, modelling most known problems in computer science.

\section{Ideas and Research Questions}

\subsection{Near Goals}

\paragraph{Question 1.} Do linear algebraic formulations of graph theoretic questions yield better performance then there combinatorial counterparts.

\paragraph{Question 2.} Is a meta-heuristic approach to subgraph isomorphism testing more efficient than the traditional combinatorial approach.

\subsection{Far Goals}

\paragraph{Question 3.} Can we make the best algorithm for large and dense subgraph isomorphism testing.

\paragraph{Question 4.} Can we make an algorithm that works on real world databases for efficient query.

\section{Note Template Section Example}

\begin{lemma}\label{lem:fake}
	This is a fake lemma
\end{lemma}

\begin{proof}
	\begin{align*}
		x^{2} + 2^{x} & = 9      \\
		5             & = 3x + 2
	\end{align*}

	By union bound and constant probability.
\end{proof}

Here is an example theorem.

\begin{theorem}\label{thm:fake}
	Fake Theorem
\end{theorem}

\begin{proof}
	By~\cref{lem:fake}
\end{proof}

Example Footnote\footnote{In Lorem Ipsum}

\end{document}
